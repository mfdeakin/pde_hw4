
\documentclass[letterpaper,11pt]{article}

\usepackage{latexsym}
\usepackage{amsmath}
\usepackage{amssymb}
\usepackage{fancyhdr}
\usepackage[margin=1.0in, left=0.5in, right=0.5in, top=1.25in, headsep=10mm, headheight=15mm]{geometry}
\usepackage{graphicx}

\pagestyle{fancy}
\rhead{Michael Deakin\\Math 516\\Homework 4}

\newcommand*\limitset[1]{{#1}^\prime}
\newcommand*\closure[1]{\overline{#1}}
\newcommand*\closureunion[1]{{#1}\cup \limitset{#1}}
\newcommand*\interior[1]{{#1}^\circ}
% The set of points within some distance #1 from #2
\newcommand*\neighbor[2]{N_{#1}({#2})}
% The neighborhood without #2
\newcommand*\delneighbor[2]{N_{#1}^*({#2})}
\newcommand*\set[1]{\{ #1 \} }
\newcommand*\conjugate[1]{\overline{#1}}
\newcommand*\sequence[2]{\set{#1}_{#2=1}^\infty}
\newcommand*\series[2]{\sum_{#2=1}^\infty #1_{#2}}
\newcommand*\compose[2]{#1 \circ #2}
\newcommand*\udisk[0]{\mathbb{D}}
\newcommand*\disk[2]{D_{#1}(#2)}
\newcommand*\punctdisk[2]{\disk_{ #1 } - \set{#2}}
\newcommand*\complex[0]{\mathbb{C}}
\newcommand*\naturals[0]{\mathbb{N}}
\newcommand*\rationals[0]{\mathbb{Q}}
\newcommand*\reals[0]{\mathbb{R}}

\newcommand*\domain[0]{\Omega}
\newcommand*\bndry[1]{\partial #1}
\newcommand*\bndrydom[0]{\partial \domain}
\newcommand*\compactcont[0]{\subset \subset} % U \compactcont V \rightarrow U \subset \closure{U} \subset V, where U, V are (open) domains

\newcommand*\ball[2]{B_{#2}(#1)}

\newcommand*\limitto[2]{\lim \limits_{#1 \rightarrow #2}}

\newcommand{\dd}[1]{\;\mathrm{d}#1}
\newcommand{\dx}{\dd{x}}
\newcommand{\dy}{\dd{y}}
\newcommand{\dz}{\dd{z}}
\newcommand{\dr}{\dd{r}}
\newcommand{\ds}{\dd{s}}
\newcommand{\dt}{\dd{t}}
\newcommand*\pderiv[2]{\frac{\partial #1}{\partial #2}}
\newcommand*\nthpderiv[3]{\frac{\partial^{#3} #1}{\partial #2^{#3}}}
\newcommand*\deriv[2]{\frac{\dd{#1}}{\dd{#2}}}
\newcommand*\nthderiv[3]{\frac{\dd{^{#3} #1}}{\dd{#2^{#3}}}}

\DeclareMathOperator{\res}{res}
\DeclareMathOperator{\sign}{sign}
\DeclareMathOperator{\diam}{diam}
\DeclareMathOperator{\partition}{Partition}

% Average integral from https://tex.stackexchange.com/questions/759/average-integral-symbol
\def\Xint#1{\mathchoice
{\XXint\displaystyle\textstyle{#1}}%
{\XXint\textstyle\scriptstyle{#1}}%
{\XXint\scriptstyle\scriptscriptstyle{#1}}%
{\XXint\scriptscriptstyle\scriptscriptstyle{#1}}%
\!\int}
\def\XXint#1#2#3{{\setbox0=\hbox{$#1{#2#3}{\int}$ }
\vcenter{\hbox{$#2#3$ }}\kern-.6\wd0}}
\def\ddashint{\Xint=}
\def\dashint{\Xint-}
\def\avgint{\dashint}

\begin{document}

\begin{enumerate}
\item Let $u \in C_c^\infty(\reals^n)$. Show that $|u(x_1, \ldots, x_n)| \leq \frac{1}{2} \int_{-\infty}^{\infty} |\partial_1 u(x_1, \ldots, x_n)| \dd{x_1}$.

\item (Gagliardo-Nirenberg interpolation inequality)
Let $n \geq 2$, $1 < p < n$, and $1 \leq q < r < \frac{n p}{n - p}$.
For some $\theta \in (0, 1)$ and some constant $C > 0$ we have
$$
||u||_{L^r(\reals^n)} \leq C ||u||_{L^q(\reals^n)}^{1 - \theta} ||D u||_{L^p(\reals^n)}^{\theta}; \forall u \in C_c^\infty(\reals^n)
$$

\begin{enumerate}
\item Use scaling to find $\theta$.
\item Prove the inequality
\end{enumerate}
Hint: Do an interpretation of $L^r$ in terms of $L^q$ and $L^{\frac{n p}{n - p}}$ and then apply Sobolev.
(nb. interpretation should be interpolation)

First we define $u_\lambda(x) = u(\lambda x)$, and note that
\begin{align*}
\int_{\reals^n} |u_\lambda|^s \dx = &\int_{\reals^n} |u(\lambda x)|^s \dx = \frac{1}{\lambda^n} \int_{\reals^n} |u(y)|^s \dy\\
\int_{\reals^n} |D u_\lambda|^s \dx = &\int_{\reals^n} |D u(\lambda x)|^s \dx = \frac{\lambda^s}{\lambda^n} \int_{\reals^n} |D u(y)|^s \dy\\
\end{align*}
Then, if we assume that the $||u||_{L^r(\reals^n)} \leq C ||u||_{L^q(\reals^n)}^{1 - \theta} ||D u||_{L^p(\reals^n)}^{\theta}$
holds for any $u \in C_c^\infty(\reals^n)$, we have
\begin{align*}
\frac{1}{\lambda^{n / r}} ||u||_{L^{r}(\reals^n)} \leq &C \frac{1}{\lambda^{(1 - \theta) n / q}} ||u||_{L^{q}(\reals^n)}^{1 - \theta} \frac{\lambda^{\theta}}{\lambda^{\theta n / p}} ||D u||_{L^p(\reals^n)}^\theta\\
  = & C \lambda^{\theta - (1 - \theta) n / q - \theta n / p} ||u||_{L^{q}(\reals^n)}^{1 - \theta} ||D u||_{L^p(\reals^n)}^\theta
\end{align*}
Then if $n / r + \theta - (1 - \theta) n / q - \theta n / p > 0$, since $C$ is independent of $u$,
any sequence with $\lambda \rightarrow 0$ will result in
\begin{align*}
C \lambda^{n / r + \theta - (1 - \theta) n / q - \theta n / p} ||u||_{L^{q}(\reals^n)}^{1 - \theta} ||D u||_{L^p(\reals^n)}^\theta \rightarrow 0
\end{align*}
which contradicts the requirement that
\begin{align*}
||u||_{L^r(\reals^n)} \leq C \lambda^{n / r + \theta - (1 - \theta) n / q - \theta n / p} ||u||_{L^{q}(\reals^n)}^{1 - \theta} ||D u||_{L^p(\reals^n)}^\theta
\end{align*}
for any non-zero $u \in L^r(\reals^n)$.
Considering $n / r + \theta - (1 - \theta) n / q - \theta n / p \leq 0$,
any sequence with $\lambda \rightarrow \infty$ will result in
\begin{align*}
C \lambda^{n / r + \theta - (1 - \theta) n / q - \theta n / p} ||u||_{L^{q}(\reals^n)}^{1 - \theta} ||D u||_{L^p(\reals^n)}^\theta \rightarrow 0
\end{align*}
causing the same contradiction.

Thus, we must have $n / r + \theta - (1 - \theta) n / q - \theta n / p = 0$; or
\begin{align*}
\theta = &\frac{\frac{n}{q} - \frac{n}{r}}{1 + \frac{n}{q} - \frac{n}{p}}\\
       = &\frac{p n}{r} \frac{r - q}{p q + n p - n q}
\end{align*}

Next, we show that we can choose a $C$ s.t. the original inequality holds when $\theta$ satisfies the previous.
By the interpolation inequality, if $\frac{1}{r} = \frac{1 - \theta}{q} + \frac{\theta}{n p / (n - p)}$,
there exists a constant $C$ s.t.
\begin{align*}
||u||_{L^{r}(\reals^n)} \leq C ||u||_{L^{q}(\reals^n)}^{1 - \theta} ||u||_{L^{\frac{n p}{n - p}}(\reals^n)}^{\theta}
\end{align*}
Happily, the relation $\frac{1}{r} = \frac{1 - \theta}{q} + \frac{\theta}{n p / (n - p)}$
holds with our choice of $\theta$ above.

Then, by the Gagliardo-Nirenberg-Sobolev inequality, since $1 \leq p < n$ and $1 \leq \frac{n p}{n - p} < \infty$,
for some other $C$ we have
\begin{align*}
||u||_{L^{\frac{n p}{n - p}}(\reals^n)} \leq &C ||D u||_{L^p(\reals^n)}\\
||u||_{L^{\frac{n p}{n - p}}(\reals^n)}^{\theta} \leq &C ||D u||_{L^p(\reals^n)}^\theta
\end{align*}
Thus,
\begin{align*}
||u||_{L^{r}(\reals^n)} \leq C ||u||_{L^{q}(\reals^n)}^{1 - \theta} ||D u||_{L^{p}(\reals^n)}^\theta
\end{align*}
completing our proof.

\item Fix $\alpha > 0$, $1 < p < \infty$, and let $\domain = B_1(0)$.
Show that there exists a constant $C$, depending on $n, p, \alpha$ s.t.
$$
\int_\domain u^p \dx \leq C \int_\domain |D u|^p \dx
$$
provided
$$
u \in W^{1, p}(\domain), |\set{x \in \domain | u(x) = 0}| \geq \alpha
$$

First we consider the case where $p < n$.
Then $p \in \left[1, \frac{n p}{n - p}\right]$, so by Theorem 5.6.3 in Evans, there exists a constant $C$ s.t.
$||u||_{L^p(\domain)} \leq C ||D u||_{L^p(\domain)}$.
This leaves us with the cases with $p \geq n$.

Assume that the claim is false; ie for any $k \in \naturals$, we can find a $u_k$
with $\lnormdom{u_k}{p} > k \lnormdom{D u_k}{p}$.
Define $v_k = \frac{u_k}{\lnormdom{u_k}{p}}$; then $\lnormdom{v_k}{p} = 1 > k \lnormdom{D v_k}{p}$.
Then $\wnormdom{v_k}{p}{1} \leq 2$, so by the compactness of $W^{k, p}(\domain)$ in $L^p(\domain)$,
there must be a subsequence $\{ v_{k_j} \}$ and $v \in L^p(\domain)$ with $\lnormdom{v_{k_j} - v} \rightarrow 0$.
Then by the dominated convergence theorem
\begin{align*}
  \int_{\domain} v \phi_{x_i} \dx = &\int_{\domain} \limitto{j}{\infty} v_{k_j} \phi_{x_i} \dx\\
    = &\limitto{j}{\infty} \int_{\domain} v_{k_j} \phi_{x_i} \dx
    = \limitto{j}{\infty} -\int_{\domain} v_{k_j, x_i} \phi \dx = 0
\end{align*}
Thus, $D v = 0$ almost everywhere, implying $v$ is constant.
Since $|\set{x | v_k(x) = 0}| \geq \alpha$, we must also have $|\set{x | v(x) = 0}| \geq \alpha >0$
Then $v = 0$, contradicting $\lnormdom{v}{p} = 1$, and thus, that the sequence
$\set{v_k | \lnormdom{v_k}{p} > k \lnormdom{D v_k}{p}}$ exists.

\item \begin{enumerate}
\item Show that $W^{1, 2}(\reals^N) \subset L^2(\reals^N)$ is not compact.

\item Let $n > 4$
  Show that the embedding $W^{2, 2}(\domain) \rightarrow L^{\frac{2 n}{n - 4}}(\domain)$ is not compact.

\item Describe the embedding of $W^{3, p}(\domain)$ in different dimensions.
  State if the embedding is continuous or compact.

\end{enumerate}

\item \begin{enumerate}
\item Let $u \in W_r^{1, 2} \subset H_r^1 = \set{u \in W^{1, 2}(\reals^n) | u = u(r)}$.
  Show that $|u(r)| \leq C ||u||_{W^{1, 2}} r^{\frac{1 - n}{2}}$.

\item Show that for $n \geq 2$, the embedding $W_r^{1, 2} \subset L^p$ is compact when $2 < p < \frac{2 n}{n - 2}$.

\item Let $u = D_r^{1, 2} = \set{\int |\nabla u|^2 \dx < \infty | u = u(r)}$.
  Show that $D_r^{1, 2} \subset L^{\frac{2 n}{n - 2}}$ and $|u(r)| \leq C ||\nabla u||_{L^2} r^{\frac{2 - n}{2}}$.
    However show that $D_r^{1, 2} \subset L^{\frac{2 n}{n - 2}}$ is not compact.
\end{enumerate}


\item Let $\domain = (-1, 1)$.
Show that the dual space of $H^1(\domain)$ is isomorphic to $H^{-1}(\domain) + E^*$
where $E^*$ is the two dimensional subspace of $H^1(\domain)$ spanned by the orthogonal vectors $\set{e^x, e^{-x}}$.

Let $u \in H_0^1(\domain)$, $v \in H^1(\domain) \cap (H_0^1(\domain))^{\perp}$,
with $(f, g) = \int_\domain u v + u' v' \dx$ for $f, g \in H^1(\domain)$.
Then $(u, v) = 0 = \int_\domain u v + u' v' \dx$.
If $v \in C^2(\domain)$, integrating by parts gives
\begin{align*}
  0 = &\int_\domain u v - u v'' \dx + u(1) v'(1) - u(-1) v'(-1)\\
    = &\int_\domain u (v - v'') \dx
\end{align*}
Then $v$ is a weak solution to the ODE $-v'' + v = 0$.

Recall that given boundary conditions, the strong form has a unique solution,
namely $v = c_1 e^x + c_2 e^{-x}$.
Since the solution to the strong form is a solution to the weak form,
and solutions to this weak form are unique, this solution is the only solution to our weak form.
Thus, if $v \in H^1(\domain)$, we must have $v = u + c_1 e^x + c_2 e^{-x}$ for some $u \in H_0^1(\domain)$,
$c_1, c_2 \in \reals$, or equivalently, $H^1 = H_0^1 + E$.

%% Then by the linearity of the dual space, $(H^1)^* = H^{-1} + E^*$.
%% Thus, they are trivially isomorphic, with the identity operator as their isomorphism.

\item \begin{enumerate}
  \item Assume that $\domain$ is connected.
  A function $u \in W^{1, 2}(\domain)$ is a weak solution of the Neumann problem
  $$
  \begin{cases}
    -\Delta u(x) = f(x) & x \in \domain\\
    \pderiv{u}{\nu} = 0 & x \in \bndrydom
  \end{cases}
  $$
  if
  $$
  \int_{\domain} D u \cdot D v \dx = \int_{\domain} f v \dx, \forall v \in W^{1, 2}(\domain)
  $$

  \item Discuss how to define a weak solution of the Poisson equation with Robin boundary conditions
  $$
  \begin{cases}
    -\Delta u(x) = f(x) & x \in \domain\\
    u + \pderiv{u}{\nu} = 0 & x \in \bndrydom
  \end{cases}
  $$

\end{enumerate}

\item \begin{enumerate}
\item Discuss the definition of weak solutions $u \in H_0^2(\domain)$ to
  $$
  \begin{cases}
    \Delta^2 u(x) = f(x) & x \in \domain\\
    u(x) = \pderiv{u}{\nu}(x) = 0 & x \in \bndrydom
  \end{cases}
  $$

  Consider $v \in C^2(\domain)$. Then $\int_{\domain} f v \dx = \int_{\domain} (\Delta^2 u) v \dx$.
  Integrating by parts and noting that $u \in H_0^2(\domain) \rightarrow D^{\alpha} u = 0$ gives
  \begin{align*}
    \int_{\domain} f v \dx = &-\int_{\domain} D(\Delta u) \cdot D v \dx
                             + \int_{\bndrydom} \left(\pderiv{}{\nu} \Delta u\right) v \dS\\
                          = &-\int_{\domain} D(\Delta u) \cdot D v \dx
                          = \int_{\domain} (\Delta u) (\Delta v) \dx
                            - \int_{\bndrydom} \Delta u \pderiv{}{\nu} v \dS\\
                          = &\int_{\domain} (\Delta u) (\Delta v) \dx
  \end{align*}
  Then any solution to the strong problem will be a solution to the above weak problem.
  We can then define the bilinear form $B[u, v] = \int_{\domain} (\Delta u) (\Delta  v) \dx$,
  with $B: H^2(\domain) \times H^2(\domain) \rightarrow \reals$.
  We can use the Sobolev Inequality and a modification of the argument for Poincare's Inequality
  to prove that this is a bounded coercive form.
  Thus, Lax-Millgram applies, so we know that there is a unique $u \in H_0^2(\domain)$ with $B[u, v] = <f, v>$.

\item Given $f \in L^2(\domain)$ prove that there exists a unique weak solution to $\Delta^2 u(x) = f(x)$.

  First
  \begin{align*}
    |B[u, v]| = &\left| \int_{\domain} (\Delta u) (\Delta v) \dx \right|\\
           \leq &\int_{\domain} |\Delta u| |\Delta v| \dx
           \leq C \lnormdom{\Delta u}{2} \lnormdom{\Delta v}{2}\\
           \leq &C \hnormdom{u}{2} \hnormdom{v}{2}
  \end{align*}
  Thus, $B$ is a bounded bilinear form.

  Next, we need to show that $B$ is coercive, i.e. that there is a constant $C$ s.t.
  $\hnormdom{u}{2} \leq C B[u, u]$.

  If $n \geq 3$, we can apply Evans 5.6.3, so
  \begin{align*}
    \hnormdom{u}{2}^2 = &\int_{\domain} u^2 \dx + \int_{\domain} |D u|^2 \dx + \int_{\domain} |\Delta u|^2 \dx\\
                   \leq & C \int_{\domain} |D u|^2 \dx + \int_{\domain} |\Delta u|^2 \dx\\
                   \leq & C \int_{\domain} |\Delta u|^2 \dx = C B[u, u]
  \end{align*}
  Otherwise, assume that there is a sequence $\set{u_k}$ with $\hnormdom{u_k}{2}^2 > k B[u_k u_k]$.
  Note that $u_k \neq 0$, as otherwise the above inequality does not hold.
  Then since $B$ is linear, we can assume that $\hnormdom{u_k}{2} = 1$.
  Then
  \begin{align*}
    0 = &\limitto{k}{\infty} \frac{1}{k} > \limitto{k}{\infty} \lnormdom{\Delta u_k}{2}^2\\
    0 = \limitto{k}{\infty} \Delta u_k = \Delta u
  \end{align*}
  Then (as shown in a previous homework) $D u$ is constant in $\domain$.
  But $D u = 0$ on $\bndrydom$, so $D u = 0$ everywhere.
  Then by the same argument, $u = 0$ everywhere, contradicting our assumption.

  Thus, $B$ is coercive and bounded, so Lax-Millgram applies,
  proving that there is a unique $u \in H_0^2(\domain)$ s.t. $B[u, v] = <f, v>$
  whenever $v \in C^\infty(\domain)$.
\end{enumerate}

\end{enumerate}
\end{document}
