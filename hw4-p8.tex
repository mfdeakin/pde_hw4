\begin{enumerate}
\item Discuss the definition of weak solutions $u \in H_0^2(\domain)$ to
  $$
  \begin{cases}
    \Delta^2 u(x) = f(x) & x \in \domain\\
    u(x) = \pderiv{u}{\nu}(x) = 0 & x \in \bndrydom
  \end{cases}
  $$

  Consider $v \in C^2(\domain)$. Then $\int_{\domain} f v \dx = \int_{\domain} (\Delta^2 u) v \dx$.
  Integrating by parts and noting that $u \in H_0^2(\domain) \rightarrow D^{\alpha} u = 0$ gives
  \begin{align*}
    \int_{\domain} f v \dx = &-\int_{\domain} D(\Delta u) \cdot D v \dx
                             + \int_{\bndrydom} \left(\pderiv{}{\nu} \Delta u\right) v \dS\\
                          = &-\int_{\domain} D(\Delta u) \cdot D v \dx
                          = \int_{\domain} (\Delta u) (\Delta v) \dx
                            - \int_{\bndrydom} \Delta u \pderiv{}{\nu} v \dS\\
                          = &\int_{\domain} (\Delta u) (\Delta v) \dx
  \end{align*}
  Then any solution to the strong problem will be a solution to the above weak problem.
  We can then define the bilinear form $B[u, v] = \int_{\domain} (\Delta u) (\Delta  v) \dx$,
  with $B: H^2(\domain) \times H^2(\domain) \rightarrow \reals$.
  We can use the Sobolev Inequality and a modification of the argument for Poincare's Inequality
  to prove that this is a bounded coercive form.
  Thus, Lax-Millgram applies, so we know that there is a unique $u \in H_0^2(\domain)$ with $B[u, v] = <f, v>$.

\item Given $f \in L^2(\domain)$ prove that there exists a unique weak solution to $\Delta^2 u(x) = f(x)$.

  First
  \begin{align*}
    |B[u, v]| = &\left| \int_{\domain} (\Delta u) (\Delta v) \dx \right|\\
           \leq &\int_{\domain} |\Delta u| |\Delta v| \dx
           \leq C \lnormdom{\Delta u}{2} \lnormdom{\Delta v}{2}\\
           \leq &C \hnormdom{u}{2} \hnormdom{v}{2}
  \end{align*}
  Thus, $B$ is a bounded bilinear form.

  Next, we need to show that $B$ is coercive, i.e. that there is a constant $C$ s.t.
  $\hnormdom{u}{2} \leq C B[u, u]$.

  If $n \geq 3$, we can apply Evans 5.6.3, so
  \begin{align*}
    \hnormdom{u}{2}^2 = &\int_{\domain} u^2 \dx + \int_{\domain} |D u|^2 \dx + \int_{\domain} |\Delta u|^2 \dx\\
                   \leq & C \int_{\domain} |D u|^2 \dx + \int_{\domain} |\Delta u|^2 \dx\\
                   \leq & C \int_{\domain} |\Delta u|^2 \dx = C B[u, u]
  \end{align*}
  Otherwise, assume that there is a sequence $\set{u_k}$ with $\hnormdom{u_k}{2}^2 > k B[u_k u_k]$.
  Note that $u_k \neq 0$, as otherwise the above inequality does not hold.
  Then since $B$ is linear, we can assume that $\hnormdom{u_k}{2} = 1$.
  Then
  \begin{align*}
    0 = &\limitto{k}{\infty} \frac{1}{k} > \limitto{k}{\infty} \lnormdom{\Delta u_k}{2}^2\\
    0 = \limitto{k}{\infty} \Delta u_k = \Delta u
  \end{align*}
  Then (as shown in a previous homework) $D u$ is constant in $\domain$.
  But $D u = 0$ on $\bndrydom$, so $D u = 0$ everywhere.
  Then by the same argument, $u = 0$ everywhere, contradicting our assumption.

  Thus, $B$ is coercive and bounded, so Lax-Millgram applies,
  proving that there is a unique $u \in H_0^2(\domain)$ s.t. $B[u, v] = <f, v>$
  whenever $v \in C^\infty(\domain)$.
\end{enumerate}
